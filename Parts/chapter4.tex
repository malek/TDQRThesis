\chapter{Conclusion et perspectives}
Le thème abordé dans ce mémoire est le traitement et analyse de données pour l'évaluation de la qualité de la route.  Nous avons commencé par montrer les différents services que nous pouvons obtenir à partir des capteurs de smartphone, et comment pouvons-nous en bénéficier dans le domaine de la détection des routes en citant plusieurs excellents travaux. Par la suite nous avons présenté l'algorithme que nous avons utilisé et comment nous l'avons lié au système qui collecte les données pour les traiter et les servir.

Ce travail reste une première version faite en utilisant les moyens relativement limités à notre disposition. Mais cela reste un bon point de départ qu’il serait intéressant à développer. Plusieurs points pourraient améliorer ce modèle, nous citons : 

\begin{itemize}
    \item Développer l'algorithme pour qu'il puisse détecter des anomalies plus spécifiques.
    \item Remplacer l'algorithme Z Score par un modèle d'apprentissage automatique.
    \item Utiliser un système plus précis pour collecter des données, comme des systèmes intégrés dans les véhicules d'état.
    \item Créer une base de données et l'associer à à l'application de la DTW pour les alerter des anomalies localisées.
    \item Faire une application pour les conducteurs, qui fournit une carte en temps réel et qui suggère la route la moins endommagée.
\end{itemize}



