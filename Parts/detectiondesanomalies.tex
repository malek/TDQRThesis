\renewcommand\labelitemi{-}
\renewcommand\labelitemii{$\circ$}
\renewcommand {\thesection}{\arabic{section}}
\chapter{Détection des anomalies de la route}

Avec cette évolution dans le domaine des mobiles, il est possible de développer un système pratique et efficace à faible coût et
 recueilli divers types d'informations afin de détecter la qualité de la surface des routes et aussi les anomalies routières telles
  que les ralentisseurs et les nids de-poule.

\section{Applications similaires}
  \subsection{MIROAD}
  Johnson et al ont proposé un système connu sous le nom de MIROAD: une plate-forme de capteurs mobiles pour
 la reconnaissance intelligente de la conduite agressive, classant le style de conduite en normal, agressif et très agressif.
Plusieurs capteurs sont utilisés tels que (accéléromètre, gyroscope, magnétomètre, GPS, vidéo) de I phone 
4 et fusionnent les données associées en un seul classificateur basé sur l'algorithme Dynamic Time Warping (DTW) qui peut détecter
 avec précision les événements avec un ensemble d'entraînement très limité (modèles). Le système détecte et enregistre activement les
  événements qui caractérisent le style d’un conducteur, augmentant ainsi la conscience d’actions potentiellement agressives et favorisant
   davantage la sécurité du conducteur.Du coup ils montrent que les capteurs disponibles dans les smartphones peuvent détecter des mouvements
    avec une qualité similaire à celle d'un bus CAN de véhicule.
  \subsection{Real time pothole detection}
  Mednis et al., ils ont proposé un système de détection des nids-de-poule en temps réel qui utilise des smartphones
 basés sur Android OS avec capteur accéléromètre. Leur système est capable de détecter les nids-de-poule en
  temps réel ainsi que les données pour le post-traitement hors ligne. Quatre algorithmes ont été proposés pour la détection 
  des nids-de-poule. Les deux premiers algorithmes (Z-THRESH et Z-DIFF) sont destinés à la détection en temps réel et les deux
   autres (STDEV (Z) et G-ZERO) sont utilisés pour le post-traitement hors ligne des données.
  \subsection{Wolverine}
  Wolverine utilise des données de capteur d'accéléromètre pour la détection des bosses et des freins.Ils ont également proposé un algorithme
 pour réorienter le téléphone mobile pour l'aligner avec l'axe du véhicule, car le téléphone peut être à n'importe quel
  endroit arbitraire à l'intérieur du véhicule. Le regroupement K-means est utilisé pour classer les données du capteur 
  en deux classes qui sont lisses ou cahoteuses (pour la détection de bosses) et freinent ou non (pour la détection de freinage).
   Support Vector Machine (SVM) est formé pour la classification des points de données pendant la phase de test pour la prédiction
    de l'état du véhicule. Ce système donne un taux de faux négatifs de 10\% pour la détection des bosses et un taux de faux
     négatifs de 21,6\% et un taux de faux positifs de 2,7\% pour la détection de freinage







\renewcommand {\thesection}{\thechapter.\arabic{section}}